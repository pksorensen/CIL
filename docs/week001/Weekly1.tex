%\documentclass[10pt,conference]{IEEEtran}
\documentclass[10pt,journal,twocolumn]{IEEEtran}
%\documentclass[10pt,technote)]{IEEEtran}
%\documentclass[10pt,conference]{IEEEtran}
%\documentclass[10pt,conference]{IEEEtran}

\usepackage[utf8x]{inputenc}
\usepackage[cmex10]{amsmath}
\interdisplaylinepenalty=2500
\usepackage{graphicx,color,psfrag}

%\usepackage{amsmath,graphicx,subfigure}
%\usepackage[T1]{fontenc}
%\usepackage{lmodern}

%\usepackage{fixltx2e}
%\usepackage{stfloats}
\usepackage{dblfloatfix} %dblfloatfix

\usepackage{url}

\DeclareFontFamily{OT1}{pzc}{}
\DeclareFontShape{OT1}{pzc}{m}{it}{<-> s * [1.10] pzcmi7t}{}
\DeclareMathAlphabet{\mathpzc}{OT1}{pzc}{m}{it}

\usepackage{cite} %However, IEEEtran pre-defines some format control macros to facilitate easy use with Donald Arseneau’s cite.sty package [13]. So, all an author has to do is to call cite.sty:

\setlength{\parskip}{1.5mm}

\def\tm{\leavevmode\hbox{$\rm {}^{TM}$}}
\def\registered{{\ooalign{\hfil\raise .00ex\hbox{\scriptsize R}\hfil\crcr\mathhexbox20D}}} % $^\registered$
\def\trademark{$\rm {}^{\hbox{\tiny TM}}$} % then accessible in math mode

\newcommand{\vect}[1]{\boldsymbol{#1}} %If bold vectors.
%\newcommand{\vect}[1]{\vec{#1}}  %If arrow over vectors.

\title{Introduction to Active Appearance Model\\  Weekly Report 1}
\author{Poul~K.~Sørensen
\thanks{Weekly Report 1}
}
%\setlength{\columnsep}{distance}


\newenvironment{listofabbrv}[1]{\begin{itemize}}{\end{itemize}}

\def\testout{
\newwrite\outputstream
\immediate\openout\outputstream=myfile.tmp
\immediate\write\outputstream{foo 1}
\immediate\write\outputstream{foo 2}
\immediate\write\outputstream{\string\textbf{foo 3}}
\immediate\closeout\outputstream
 }


\newif\ifabrvused
\abrvusedfalse

\makeatletter 
\def\refabrv#1{\abrv{#1}{}}

\def\abrv#1#2{%
  \ifabrvused\else%
    \newwrite\@loa%
    \immediate\openout\@loa=\jobname.loa%
    \global\abrvusedtrue%
  \fi%  
  \@ifundefined{abrv@#1}{%
    \expandafter\def\csname abrv@#1\endcsname{#2}%
    \immediate\write\@loa{\unexpanded{\item[\textbf{#1}]#2}}%
    #2 (\textbf{#1})%
  }{%
  {\textbf{#1}}%
  }%
}

\newcommand\@startloa{%
  \immediate\closeout\@loa
  \section*{Abbreviations}
  \renewcommand{\IEEEiedlistdecl}{\IEEEsetlabelwidth{SONET}}
  \begin{listofabbrv}{SPMD}
  \InputIfFileExists{\jobname.loa}{}{\item[\null]}
  \end{listofabbrv}
  \renewcommand{\IEEEiedlistdecl}{\relax}% reset back
%  \newwrite\@loa
%  \immediate\openout\@loa=\jobname.loa
}
\newcommand{\listofabbreviations}{\@startloa}

% abvr{abvr}{text}
\makeatother

\newcommand{\ie}{i.e.~\/}
\newcommand{\eg}{e.g.~\/}
\newcommand{\degree}{$^{\circ }$}

\newcommand{\cpp}{\texttt C++~}
\newcommand{\cppcx}{\texttt C++/CX~}

%\usepackage{acronym}
\IEEEoverridecommandlockouts %The following commands are intentionally disabled: \thanks,\IEEEPARstart, \IEEEbiography, \IEEEbiographynophoto, \IEEEpubid, \IEEEpubidadjcol, \IEEEmembership, and \IEEEaftertitletext. If needed, they can be reenabled by issuing the command:

%\setlength{\IEEEilabelindent}{1cm}

\begin{document}
\maketitle
%\IEEEdisplaynotcompsoctitleabstractindextext{ %
%\IEEEcompsoctitleabstractindextext

\section{Introduction}
\IEEEPARstart{A}{ctive Appearance Models} have been used for over a decade and first introduced back in 1998. It was introduced as a rapid, accurate and robust algorithm for finding deformable objects in imaging. 

The problem was noticed in a presentation given by Henning Kabel at a Microsoft Developer Event, where the user manually had to position the image into the silhouette to get the application to work correctly. The application was making it possible for the user to change the hair style virtually and during the presentation, another interesting vision topic was introduced. The problem of adding skin colour to areas where hair had been removed. This problem wont be addressed here, but having a proper facial model telling where eyes, nose, mouth, jaw and eye brows are located, it automatic becomes easier to know where to sample skin colour from.

The motivation by this project is to learn how to create a library that can be used together with the Windows Runtime API for Windows Store Applications and solving the problem of aligning an image of a face correctly into a face silhouette by \abrv{ML}{Machine Learning}. 

Creating a library in \cpp makes it easy to port or use from other mobile platforms, but it leaves some concerns about compatibility with the ARM-architecture. Cross compilers exists for \cpp and the real issue is about external libraries and getting them cross compiled. This is also the reason for creating a library and not using existing solutions. In this project the goal will be to get a working solution for the x86 architecture and if possible make it work for ARM also, but this will be depending on external libraries. 

\begin{table*}[!b]
\normalsize
\centering
\renewcommand{\arraystretch}{1.3}
\caption{Time Schedule}
\begin{tabular}{lllll}
Project Week 	& Date 	& Hours & Activity 											& Risk	\\ \hline
0 				& 1/3	& 8		& Write a plan.										& 1		\\
1				& 7/3   & 8     & PCA and Procrustes Analyse						& 1 	\\
2				& 14/3  & 8	    & Prior Implementations Study 						& 1     \\
3				& 28/1  & 8     & External Libraries Study							& 1		\\
4				& 1/3 	& 11	& 													& 3		\\
5				& 11/2	& 11	& Implementation									& 3		\\
6				& 18/2	& 11	& Implementation									& 3		\\
7				& 25/2  & 11	& Tests (Compare with current Matlab Implementation)& 1		\\
8				& 4/3   & 11  	& Report Work										& 1		\\ \hline
\multicolumn{2}{l}{Total}& \multicolumn{3}{l}{111}\\ \hline
Deadline  		& 11/3  & 	& Turn in Code and Report/Journal.					& 0		\\
\end{tabular}

\label{tab:timeschedule}
\hrulefill
\vspace*{4pt}
\end{table*}

\section{Time Schedule and Risk Analysis}
The project is a 5 ECTS course\footnote{13 weeks of 8 hours} of approx. 110 hours work. In the following tree weeks half of the hours will be used for literature studies and followed by 5 weeks of implementing and reporting. In the plan the risk is classified using a scale from 1 (no risk) to 5 (high risk). The risk is described as the chance of the activity being delayed.

There is only a risk of getting delayed in the implementation state due to potential bugs or issues when coding. Hopefully the study part will not take up as much time as planned and giving a few extra hours for coding and some work have already been made into what external linear algebra libraries to use. It is planned to make a minor report or journal about the work and the theory behind the solution and possible an oral examination. 



%\section*{Abbreviations}
%\renewcommand{\IEEEiedlistdecl}{\IEEEsetlabelwidth{SONET}}
%\begin{acronym}
%\acro{acr:opencl}[\textbf{OpenCL}]{Open Computing Language}
%\acro{acr:cuda}[\textbf{CUDA\trademark}]{Compute Unified Device Architecture}
%\acro{acr:rf}[\textbf{RF}]{Random Forest}
%\end{acronym}
%\renewcommand{\IEEEiedlistdecl}{\relax}% reset back

\listofabbreviations


%\bibliographystyle{IEEEtran}
%\bibliography{IEEEabrv,bibtex}
\end{document}